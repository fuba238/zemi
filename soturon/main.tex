% !TEX encoding = UTF-8 Unicode

\documentclass[10pt,a4j]{ltjsarticle}
\usepackage{url}

\begin{document}
%表紙
\begin{titlepage}
  \begin{center}  
    \vspace*{20truept}    
    {\LARGE 2022年度 卒業論文}     
    \vspace*{75truept}    
    {\Huge Scratchからの移行を意識した} %論文タイトル
    \vspace{10truept}
    {\Huge プログラミング言語の開発} %論文タイトル 長い場合 改行1
    \vspace{10truept}
    {\Huge } %論文タイトル 改行2
    \vspace{85truept}
    {\LARGE 指導教員 須田 宇宙 准教授} 
    \vspace{60truept}
    {\LARGE 千葉工業大学 情報ネットワーク学科}
    \vspace{15truept}
    {\LARGE 須田研究室}
    \vspace{70truept}
    {\LARGE 1932034 氏名 川原 史哉 } % 氏名は消さない 学生番号 氏名 
    \vspace{70truept}
  \end{center}
  \begin{flushright}
    {\LARGE 提出日 2023年1月17日}
  \end{flushright}
\end{titlepage}

%目次
\setcounter{tocdepth}{3} %目次の深さ設定
\setcounter{page}{0} %ページ番号を0から始める
\tableofcontents %目次の作成
\thispagestyle{empty} %ページスタイルの変更
\clearpage %改ページ

\section{緒言}
%背景
近年の急速な情報化に伴い,2020年度から小学校ではプログラミング教育の必修化が行われた.これにより小学生でも扱いやすいビジュアルプログラミング言語が注目されている.
代表的なビジュアルプログラミング言語に「Scratch」がある.「Scratch」は指示の書かれたブロックを並べることで視覚的にわかりやすく,直感的な操作でプログラミングができる.また,文字のタイピング量が少ない点や英単語の知識を必要としない点などから,初学者のプログラミング学習に適している.

%問題点
本格的なプログラムを作成していくうえで,テキストプログラミング言語の学習が必要になる.しかし,Scratchに慣れていると,テキストプログラミング言語の学習を始めようとしたときに仕様の違いが挫折の要因になり得ると考えた.例えば,スプライトの有無やソースコードの記述方法などが挙げられる.

%先行研究
当研究室ではこれまで,上記の問題点を考慮したテキストプログラミング言語に関する先行研究が行われてきた.内堀は,ビジュアルプログラミングとテキストプログラミングについての調査を行い,初学者向けのテキストプログラミング言語の基礎を示した\cite{senkou1}.また,菅原は,開発するプログラミング言語の言語仕様を定め,インタープリタの開発を行った\cite{senkou2}.

しかし,先行研究の問題点として,開発するプログラミング言語の言語仕様はまとまりきっておらず,インタープリタのパーサーが正しく動作しないなどがある.

%目的
そこで本研究では,先行研究で開発されたプログラミング言語「Chiba-lang」の言語仕様をまとめることを目的とする.
\clearpage

\section{小学校でのプログラミング教育について}
\subsection{概要}
社会の情報化の進展により,コンピュータは人々の生活と密接に関わっており,様々な場面で活用されている.
このような社会での職業生活や家庭生活,余暇生活などのあらゆる活動において,情報や情報技術を活用する能力が必要となる.

また,子どもたちが将来どのような職業になるかに関わらず,「プログラミング的思考」は必要となる.

以上の背景から,文部科学省は,学習指導要領改定において,小・中・高等学校を通じてプログラミング教育を充実することとし,2020年度から小学校においてプログラミング教育の導入を決めた\cite{pdf1}.

\subsection{プログラミング的思考}
「プログラミング的思考」とは,小学校のプログラミング教育において情報活用能力に含まれる育成されるべき資質・能力の一つで,文部科学省(2020)では,
\begin{quote}
自分が意図する一連の活動を実現するために,どのような組み合わせが必要であり,一つ一つの動きに対応した記号を,どのように組み合わせたらいいのか,記号の組み合わせをどのように改善していけば,より意図した活動に近づくのか,といったことを論理的に考えていく力
\end{quote}
としており,プログラムの働きやよさ,情報社会が情報技術によって支えられていることに気づかせることを目的としている\cite{pdf1}.
\subsection{ビジュアルプログラミング言語}
ビジュアルプログラミング言語とは,視覚的なオブジェクトでプログラミングをするプログラミング言語である.直感的な操作性や視覚的なわかりやすさから初学者のプログラミング教材として適している.
指示の書かれたブロックを組み合わせるブロックタイプ,フローチャートのように,指示や機能,条件などのアイコンを線を繋ぐフロータイプ,独自のルールで作られた独自ルールタイプの三つに大別される.

ビジュアルプログラミングで代表的なものとしてScratchがあり,小学校のプログラミング教育でも活用されている.
\subsection{テキストプログラミング言語}
テキストプログラミング言語とは,文字のみでソースコードを記述するプログラミング言語である.現在のソフトウェア開発では,テキストプログラミング言語が用いられている.

ビジュアルプログラミング言語と比較すると,直感的な操作性や視覚的なわかりやすさがないため,初学者にとっては難しく感じる可能性があると考えられる.
\clearpage

\section{Chiba-langのコンセプト}
\subsection{概要}
\subsection{参考言語}
\subsubsection{Scratch}
\subsubsection{JavaScript}
\subsubsection{Ruby}
\subsubsection{Smalltalk}
\subsubsection{Elm}
\subsubsection{Lisp}
\subsubsection{Elixir}
\subsection{先行研究}
\clearpage

\section{言語仕様}
\subsection{概要}
\subsection{データ型}
\subsection{演算子}
\subsection{変数}
\subsection{関数}
\subsection{繰り返し処理}
\subsection{条件分岐}
\subsection{配列}
\subsection{Scratchとの比較}
\clearpage



\section{結言}
\clearpage

\section{謝辞}
\clearpage

\begin{thebibliography}{99}
\bibitem{senkou1} 内掘美幸: ``Scratchからの移行を考慮したプログラミング言語と環境の開発'', 2019年度卒業研究
\bibitem{senkou2} 菅原直輝: ``Scratchからの移行を考慮したプログラミング言語の開発'', 2020年度卒業研究
\bibitem{pdf1} 文部科学省,``小学校プログラミング教育の手引(第三版)'', (2020),\url{https://www.mext.go.jp/content/20200218-mxt_jogai02-100003171_002.pdf}
\end{thebibliography}

\end{document}
